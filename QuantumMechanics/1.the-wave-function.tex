\chapter{The WAVE FUNCTION}

\section{The \schrodinger{} Equation}

Position$$x(t)$$
Newton's law$$m\secd{x}{t}=-\fstpd{V}{x}$$
Particle's wave function$$\Psi(x,t)$$
\schrodinger{} equation$$i\hbar \secpd{\Psi}{t}=-\frac{\hbar^2}{2m}\secpd{\Psi}{x}+V\Psi$$

\section{The Statistical Interpretation}

Born's statistical interpretation
\begin{equation*}
	\int_{a}^{b}\left|\Psi(x,t)\right|\diff x = \left\lbrace\text{ probability of finding the particle between a and b, at time t }\right\rbrace 
\end{equation*}

The statistical interpretation introduces a kind of indeterminacy into quantum mechanics.

There are three plausible answers to this question, and they serve to characterize the main schools of thought regarding to quantum indeterminacy.

\begin{itemize}
	\item The realist position
	\item The orthodox position
	\item The agnostic position
\end{itemize}

\section{Probability}

\subsection{Discrete Variables}

\begin{equation*}
	\langle f(j)\rangle = \frac{\Sigma j N(j)}{N}=\sum_{j=0}^{\infty}f(j)P(j)
\end{equation*}

\begin{equation*}
\Delta j = j-\langle j\rangle
\end{equation*}

\begin{equation*}
\sigma^2=\langle (\Delta j)^2 \rangle = \langle j^2 \rangle - \langle j \rangle ^2
\end{equation*}

\subsection{Continuous Variables}

\begin{equation*}
\rho(x)\diff x = \left\lbrace\text{ probability that an individual (chosen at random) lies between $x$ and $x+\diff x$ }\right\rbrace 
\end{equation*}

\begin{equation*}
\langle f(x)\rangle = \int_{-\infty}^{+\infty} f(x)\rho(x)\diff x
\end{equation*}

\begin{equation*}
	\sigma^2=\langle (\Delta x)^2 \rangle = \langle x^2 \rangle - \langle x \rangle ^2
\end{equation*}

\section{Normalization}

\begin{equation*}
\int_{-\infty}^{+\infty}\left| \Psi(x,t) \right|^2 \diff x = 1
\end{equation*}